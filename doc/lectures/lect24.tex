\section{Лекция 24 (26.04.25)}

\subsection{Моделирование турбулентности методом LES}
Уравнение моментов при использовании LES модели
\begin{equation*}
\dfr{u}{t} + \vec u\cdot\nabla u = \nabla \cdot \left(\nu + \nu^{SGS}\right)\nabla u - \dfr{p}{x},
\end{equation*}

Вычисление подсеточной вязкости
\begin{equation}
\label{eq:nu_sgs}
\nu^{SGS}=l^2 \sqrt{\sum_{i,j}2 e_{ij} e_{ij}},
\end{equation}
где компоненты тензора скорости деформации определяется как
\begin{equation*}
e_{ij} = \frac12\left(\dfr{u_i}{x_j} + \dfr{u_j}{x_i}\right).
\end{equation*}

Вычисление приведённого масштаба подсеточной турбулентности $l$ согласно модели Смагоринского:
\begin{equation}
\label{eq:nu_sgs_smagor}
l_s = C_s h, 
\end{equation}
где $C_s = 0.173$, а $h$ -- характерный размер ячейки.

Аналогичное уравнение теплопроводности
\begin{equation*}
\dfr{T}{t} + \vec u\cdot\nabla T = \nabla \cdot \left(a + a^{SGS}\right)\nabla T.
\end{equation*}
В простейшем случае подсеточную температуропроводность можно вычислять через турбулентное число Прандтля
\begin{equation*}
{\rm Pr}_t = \frac{\nu^{SGS}}{a^{SGS}} \approx 1.
\end{equation*}


\subsubsection{Задание для самостоятельной работы}
В тесте \cvar{[convection-fvm]} из файла \ename{convection_fvm_les_test.cpp}
реализована SIMPLE схема решения задачи о естественной конвекции в замкнутой двумерной прямоугольной области \cref{eq:natural_convection_y}.
Высота области равна единице. Длина определена параметром \cvar{L}.
Нижняя стенка подогревается с температурой $T=1$, а верхняя остужается с $T=0$. Боковые стенки теплоизолированы.
Расчёт продолжается до момента установления стационарного решения.

Необходимо
\begin{itemize}
\item увеличив количество итераций получить установившееся решение задачи при числах $\Ren=\Pen=1000$ и $L=2$.
      С помощью анимации полей скорости и температуры проилюстрировать установление стационарного режима.
\item Провести исследование зависимости вида полученного решения от ширины области $0.3 < L < 10$.
      Узнать, при каких $L$ решение сходится. Построить график количества итоговых вихревых ячеек от ширины $L$.
\item Добавить в программу LES метод моделирования турбулентности и сравнить полученный результат с расчётом по ламинарной модели.
      Вычисление подсеточной вязкости проводить по формуле \cref{eq:nu_sgs,eq:nu_sgs_smagor}.
      Подсеточную теплопроводность $a^{SGS}$ вычислять исходя из ${\rm Pr}_t = 1$.
\item Провести один LES расчёт на нерегулярной (pebi) сетки и сравнить ответ с исходной структурированной сеткой.
\end{itemize}

Рекомендации:
\begin{itemize}
\item Для программаирования LES необходимо внести изменения в процедуру \cvar{assemble_nu}, которая вычисляет
      коэффициенты диффузии для уравнений моментов и теплопроводности на граниях сетки.
\item Для вычисления тензора скоростей деформации $\vec E$ необходимо вычислить $\dfr{u_i}{x_j}$ в центрах ячеек по аналогии с заданием \ref{sec:ke_prog}.
\item Подсеточную вязкость на граничных граниях положить нулю, на внутренних -- полусумме подсеточных вязкостей соседних ячеек.
\item Характерный размер ячейки $h$ в двумерном случае следует вычислять как квадратный корень из площади ячейки,
      которую в свою очередь можно получить процедурой \cvar{Grid::cell_volume}.
\end{itemize}
