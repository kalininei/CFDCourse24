\section{Лекция 16 (23.03)}

\subsection{Условие устойчивости}
\begin{equation}
\label{eq:femtvd_tau}
1 + \tau(1 - \theta)\min_i\frac{l_{ii}}{m_i} \geq 0.
\end{equation}
\subsection{FEM-TVD алгоритм}
TODO

\begin{equation}
\begin{array}{l}
P_i^{+} = \sum\limits_{j\neq i}\min\{0, k_{ij}\}\min\{0, u_j - u_i\}\\
P_i^{-} = \sum\limits_{j\neq i}\min\{0, k_{ij}\}\max\{0, u_j - u_i\}\\
Q_i^{+} = \sum\limits_{j\neq i}\max\{0, k_{ij}\}\max\{0, u_j - u_i\}\\
Q_i^{-} = \sum\limits_{j\neq i}\max\{0, k_{ij}\}\min\{0, u_j - u_i\}\\
r_i^{+} = Q_i^{+} / P_i^{+},\\
r_i^{-} = Q_i^{-} / P_i^{-}
\end{array}
\end{equation}
Чтобы избежать деления на ноль в случае $P_i^{\pm} = 0$, в этих случаях его можно положить малому числу.

\begin{equation}
d_{ij} = d_{ji} = \max\{0, -k_{ij}, -k_{ji}\}.
\end{equation}

\begin{equation}
l_{ij} = k_{ij} + d_{ij}
\end{equation}

\begin{equation}
\label{femtvd_fa}
f_{ij}^a = f_{ji}^a = \begin{cases}
-\min\{ F(r_i^+) d_{ij}, l_{ji} \}, &\text{если } u_i \geq u_j,\\
-\min\{ F(r_i^-) d_{ij}, l_{ji} \}, &\text{если } u_i < u_j,
\end{cases},
\qquad f^a_{ii} = -\sum\limits_{i\neq j} f^a_{ij}.
\end{equation}
$F(r)$ -- функция-ограничитель \cref{eq:tvd_limiter}.

\subsection{Задание для самостоятельной работы}
В тестовом примере \cvar{[transport2-fem-upwind-explicit]}
из файла \ename{transport-fem-solve-test.cpp}
реализовано решение двумерного уравнения переноса 
методом конечных объёмов с явной алгебраической
противопотоковой схемой.

В этом тесте используя явную схему необходимо
\begin{enumerate}
\item Реализовать TVD-схему с ограничителями upwind, minmod и superbee;
\item Создать сравнительную анимацию численного решения, полученного этими тремя схемами;
\item Нарисовать норму отклонения точного решения от численного в зависимости от времени для трёх схем.
      В качестве нормы рассмотреть среднеквадратичное отклонение и отклонение максимума от единицы;
\item Изучить влияние шага по времени на точность результата на финальный момент времени ($t=0.5$).
      Для этого необходимо провести серию расчётов с шагами по времени $\tau = k^\tau \tau_r$,
      где $\tau_r$ -- это максимально допустимый условием \eqref{eq:femtvd_tau} шаг по времени,
      а $k^\tau$ -- варьируемый коэффициент, изменяющийся от $0.1$ до тех пор, пока решение остаётся устойчивым (может быть и большим единицы).
      По результатом этих расчётов нужно для трёх схем (upwind, minmod, superbee) построить графики нормы отклонения от параметра $k^\tau$. 
\end{enumerate}


Вычиление матрицы антидиффузии $F^a$ по формулам \eqref{femtvd_fa} необходимо вести в цикле по упорядоченным связям $ij$ (\cvar{_edges}).
Для быстрого доступа к коэффициентам разреженной матрицы, связянным с текущей связью,
необходимо использовать поля \cvar{..._addr} структруры \cvar{Edge}.
Процедура заполнения матрицы $F^a$ в цикле по граням должна иметь примерно такой вид
\begin{cppcode}
CsrMatrix Fa(_fem.stencil());   // заполнить шаблон нулевыми значениями
for (const Edge& e: _edges}{
    // compute
    double fa_ij = ...

    // assign
    Fa.vals()[e.ij_addr] =  fa_ij;
    Fa.vals()[e.ii_addr] -= fa_ij;
    Fa.vals()[e.ji_addr] =  fa_ij;
    Fa.vals()[e.jj_addr] -= fa_ij;
}
\end{cppcode}
