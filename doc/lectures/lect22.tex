\section{Лекция 22 (12.04.25)}

\subsection{$k-\omega$ модель турбулентности}

Введём параметр удельной диссипации кинетической энергии
\begin{equation}
\label{eq:turb_omega}
\omega = \frac{\epsilon}{C_\mu k}.
\end{equation}

Замыкающую модель турбулентности в терминах $k-\omega$ запишем согласно (Wilcox, 1988):
\begin{align}
\label{eq:komega}
&\dfr{k}{t} + \nabla\cdot(\vec u k) - \nabla \cdot \left(\nu + \frac{\nu_T}{\sigma_k}\right)\nabla k = G - C_\mu \omega k, \\
&\dfr{\omega}{t} + \nabla\cdot(\vec u \omega) - \nabla \cdot \left(\nu + \frac{\nu_T}{\sigma_\omega}\right)\nabla \omega = 
	\frac{\gamma G}{\nu_T} - \beta \omega^2
\end{align}

\begin{equation}
G = 2 \nu_T {\rm dev} \vec E : \nabla\vec u = \nu_T \sum_{i,j}\left(\dfr{u_i}{x_j} + \dfr{u_j}{x_i}\right)\dfr{u_i}{x_j}
\end{equation}

\begin{equation}
\nu = \frac{1}\Ren, \qquad \nu_T = \frac{k}{\omega}, \qquad \nu_{eff} = \nu + \nu_T.
\end{equation}

\begin{equation*}
C_\mu = 0.09, \quad \sigma_k = 2.0, \quad \sigma_\omega = 2.0, \quad \beta = \frac{3}{40}, \quad  \gamma = \frac59
\end{equation*}

Граничные условия:
\begin{equation}
x\in\partial\Omega: \dfr{k}{n} = 0
\end{equation}

Для приграничных ячеек
\begin{equation}
i \in bnd: \omega_i = \begin{cases}
\dfrac{C_\mu^{-1/4} k^{1/2}}{\kappa y}, &\quad y^+ > 11.25, \\[10pt]
\dfrac{6 \nu}{\beta y^2},               &\quad y^+ < 11.25.
\end{cases}
\end{equation}
где $\kappa=0.41$, $y$ -- расстояние от границы, а $y^+$:
\begin{equation*}
y^+ = \frac{y u_\tau}{\nu}, \quad u_\tau = \sqrt{C_\mu^{1/2}k},
\end{equation*}

По аналогии с програмной реализацией модели $k-\epsilon$ диссипативные слагаемые (слагаемые с отрицательным знаком) из
правых частей уравнений \cref{eq:komega} следует перенести в левую часть. Линеаризация этого слагаемого осуществляется
за счёт использования множителя $\omega$ с предыдущей итерации.
Для избегания деления на ноль
вихревую вязкость будем вычислять по формуле \cref{eq:nut_nozerodiv},
где длина смешения выразиться как $l^* = \min(1, k ^{1/2} \omega^{-1})$.

\subsubsection{Задание для самостоятельной работы}
Провести ту же работу, что и при анализе модели $k-\omega$.
В тесте \ename{[cavity-fvm-rans-ko]} из файла \ename{[cavity_fvm_rans_test.cpp]}
реализована SIMPLE схема решения задачи о стационарном течении в каверне
по турбулентной модели $k-\omega$.
При для некоторых методов (помеченных комментарием \cvar{TODO})
стоит фиктивная реализация.
Необходимо
\begin{itemize}
\item Реализовать все эти методы;
\item Отрисовать итоговые поля, характеризующие скорость течения и его турбулентные характеристики;
\item Провести анализ как константы модели влияют на уровень турбулентности в каверне. Уровень турбулентности
      считать как
\begin{equation*}
S = \int\limits_\Omega k \, d\vec x.
\end{equation*}
\item Кроме того следует сравнить решения, полученные для этой задачи
с решением для $k-\epsilon$ модели.
Энергию $k$ сравнивать напрямую,
а диссипацию -- путём приведения к единому виду ($\epsilon$ или $\omega$) 
по формуле \cref{eq:turb_omega}.
\end{itemize}
